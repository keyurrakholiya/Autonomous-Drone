%%%%%%%%%%%%%%%%%%%%%%%%%%%%%%%%%%%%%%%%%%%%%%%%%%%%%%%%%%%%%%%%%%%%%%%%%%%%%
%	e-Yantra, IIT-Bombay

%	Document Author: Saurav Shandilya
%	Date: 16-August,2012
%	Last Editted by: Saurav
%   Date Last updated: 31-05-2016 

%%%%%%%%%%%%%%%%%%%%%%%%%%%%%%%%%%%%%%%%%%%%%%%%%%%%%%%%%%%%%%%%%%%%%%%%%%%%%

\documentclass[11pt,a4paper]{article}
\usepackage{graphicx}
\usepackage{hyperref}
\usepackage{graphicx}
\usepackage{caption}
\title{Drone -Test Flight 1}
\author{Keyur Rakholiya}
\date{\today}

\begin{document}
	\maketitle
	\newpage
	\tableofcontents
	\newpage
	\section{Tutorial Name}
	Drone test flight 1
	
	\section{Hardware Requirement}
	Assembled Drone
	RC remote control
	FUll charged Battery
	\begin{center}
		\includegraphics[width=300px]{Drone}
		\captionof{figure}{Done ready to fly}
	\end{center}
	

	\newpage
	\section{Theory and Description}
		before fly, always pay attention to safety. go in open ground. stay away when you trying to arm it.\newline
		
		\url{https://youtu.be/oRvV-g8zlzI}\newline
		
		we got crash in very first flight. later on we find out that , when try to give it forward motion it will actually move backward. so always check your parameters before fly. 
		change RC2 parameter to -1. default RC2 parameter is 1.
		
		\includegraphics[width=300px]{Drone2}
		\captionof{figure}{On field}
		
	\section{References}
	\url{https://github.com/eYSIP-2016/Autonomous-Drone}
	
\end{document}



