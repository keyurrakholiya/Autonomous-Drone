%%%%%%%%%%%%%%%%%%%%%%%%%%%%%%%%%%%%%%%%%%%%%%%%%%%%%%%%%%%%%%%%%%%%%%%%%%%%%
%	e-Yantra, IIT-Bombay

%	Document Author: Saurav Shandilya
%	Date: 16-August,2012
%	Last Editted by: Saurav
%   Date Last updated: 31-05-2016 

%%%%%%%%%%%%%%%%%%%%%%%%%%%%%%%%%%%%%%%%%%%%%%%%%%%%%%%%%%%%%%%%%%%%%%%%%%%%%

\documentclass[11pt,a4paper]{article}

\usepackage{graphicx}
\title{Manual flight using Raspberry pi}
\author{keyur Rakholiya}
\author{akshit gandhi}
\date{\today}

\begin{document}
	\maketitle
	\newpage
	\tableofcontents
	\newpage
	\section{Tutorial Name}
		Manual Flight using Raspberry pi

	\section{Hardware Requirement}
		Drone with APM2.6 Controller\\
		Raspberry Pi B+

	\section{Software Requirement}
		Mission planner software\\
		Moboxterm for Raspberry pi
	\section{Theory and Description}
		we have connected R pi to Apm through UARTs. we have also connect GPIO of Rpi to INPUT of APM.\\
		so we have two type of communication.\\
		1)through UARTS\\
		2)through GPIO(one way communication)\\
		\\
		we actually get feed back from APM through UART port. we are generating PWM Signal through R pi's GPIO.
		so we have removed remote(RC control) input from APM.
	\section{Experiment}
		now, we have one indoor flight in which ,we can control drone through our key board keys.\\
		we have noticed that its very stable and more accurate then remote control.
	\section{Exercise}
		whole python code for raspberry pi is given on github.\\
		\url{https://github.com/eYSIP-2016/Autonomous-Drone}

	
\end{document}



